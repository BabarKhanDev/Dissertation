\documentclass[11pt]{article}
\usepackage{url}

\title{Predicting Memorable Regions Of Images Using Deep Learning and Adversarial Networks}
\author{Babar Khan}

\begin{document}

\maketitle

\newpage{}

\section{Introduction}

\subsection{Background}

\subsubsection{We see a lot of images}

We are constantly surrounded by imagery, on the way to work, on the internet, on TV, in stores, etc. Some of them stick out more than others, we see hundreds, if not thousands, of images a day, and yet culturally and individually we all remember the same ones.

\subsubsection{Cultural vs Individual Memorability}

Due to cultural significance an image can become memorable. The 2015 dress \cite{BBCDress2015}, a country's flag, or the 1932 image “lunch atop a skyscraper” \cite{gambino_2012}, come to mind. I would argue that the memorability of these images is tied to the culture surrounding them, not necessarily due to intrinsic properties within. The focus of this research is on the intrinsic properties within an image that make it memorable to an individual, like seeing an advert on a bus and then later recognising the same advert online.

\subsubsection{How do we measure the memorability of an image}

% What work has shown independence of memorability from viewer?
% Isola et al in 2011
% Vischema dataset experiments

It has been shown that the memorability of an image is an intrinsic property, independent from the viewer \cite{Isola2011, IsolaParikhTorralbaOliva2011, ICCV15_Khosla, isola2014memorability}. This was achieved by performing a memorability game where participants are presented with a stream of images, and on an interval shown an image that they had already seen. Most participants were able, or unable, to remember the same images.
%I need to check if this is accurate for each of those sources ^^^^%
How often an image was recognised is proportional to its memorability. %Verify%
This is taken further by Akagunduz et al. \cite{VischemaPaper} where, in a similar experiment, they measure which regions of images are memorable.

\subsubsection{Why do we want to create a memorable image}

\subsubsection{How does this research help to create a memorable image}

\newpage{}

\bibliographystyle{IEEEtran.bst}
\bibliography{refs}

\end{document}