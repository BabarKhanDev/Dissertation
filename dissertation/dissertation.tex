\documentclass{UoYCSproject}
% Preamble
\usepackage{url}     % This is needed for IEEE referencing
\usepackage{bookmark}
\usepackage{algpseudocode} % Pseudocode algorithm
\usepackage{algorithm}
\usepackage[backend=biber]{biblatex}
\addbibresource{references.bib}

% Title Page
\author{Babar Khan}
\title{Predicting Memorable Regions Of Images Using Deep Learning and Adversarial Networks}
\date{Version 1, 2023-April-11}
\supervisor{Adrian Bors}
\SWE

% Document
\begin{document}
\pagenumbering{roman}
\maketitle

\tableofcontents

\chapter{Abstract}

% This needs majorly tidying up once I have the final results.

%The purpose of the research (what's it about and why's that important)
I investigate a range of conditional image to image generative models to learn which regions of a given image are memorable.
%The methodology (how you carried out the research)
This was conducted by training a variety of models to predict memorability maps given the respective image from the VISCHEMA dataset.
%The key research findings (what answers you found)
I have found that Autoencoders, even with similar network structures to GANs and Diffusion models, are unable to learn a very accurate model and seem to smooth out the blocky VMS shape into circular shapes that fade towards the edges. My autoencoder model was able to achieve a mean L1 loss of ~0.132.
GAN models can be difficult to work with, require lots of hyperparameter tuning, and if you use a traditional training loss, such as L1 or MSE, it can be impossible to see if they have converged. Lots of research has gone into creating versitile and robust GAN architectures, picking one of those and adapting it to your needs can reduce your development time considerably. Using a modified pix2pix gan models I was able to generate VMS maps far more accurately than with an autoencoder model, they maintain the blocky shape of the regions and achieve a higher mean L1 loss of ~0.0740.
% This is subject to change
Diffusion models take longer to train than GAN and autoencoder models, while they can produce high quality images [Maybe cite Dalle-2, imagen, others], the training loop requires a much larger amount of computational resources than an autoencoder or GAN does, making it quite infeesible for most people to train diffusion models for their own image distributions. 
%The implications of these findings (what these answers mean)

\chapter{Executive Summary}

%my own experience 1/2 pages separate count? Not sure really

\chapter{Introduction}

% This should be 2 pages
% need to define well what i am doing

\section{Background}

%We see a lot of images

We are constantly surrounded by imagery, on the way to work, on the internet, on TV, in stores, etc. Some images stick out more than others, we see hundreds, if not thousands, of images a day, and yet culturally and individually we all remember similar ones.

%Cultural vs Individual Memorability

Due to cultural significance an image can become memorable. The 2015 dress \cite{BBCDress2015}, a country's flag, or the 1932 image “lunch atop a skyscraper” \cite{gambino_2012}, come to mind. I argue that the memorability of such images is tied to the culture surrounding them, not necessarily due to intrinsic properties within. The focus of this research is on the intrinsic properties within an image that make it memorable to an individual, like seeing an advert on a bus and then later recognising the same advert online.

%How do we measure the memorability of an image

% What work has shown independence of memorability from viewer?
% Isola et al in 2011
% Vischema dataset experiments

It has been shown that the memorability of an image is an intrinsic property, independent from the viewer \cite{Isola2011, IsolaParikhTorralbaOliva2011, ICCV15_Khosla, isola2014memorability}. This was achieved by performing a memorability game where participants are presented with a stream of images, and on an interval shown an image that they had already seen. Most participants were able, or unable, to remember the same images.
%I need to check if this is accurate for each of those sources ^^^^%
How often an image was recognised is proportional to its memorability. %Verify%
This is taken further by Akagunduz et al. \cite{VischemaPaper} where, in a similar experiment, they measure which regions of images are memorable.

%Why do we want to create a memorable image
%How does this research help to create a memorable image

\section{Motivation}

% Why do we want to do this
% Who will this be useful for
% What motivates me

%What are my goals}

%   To create an accurate mapping from images to memorable images
%   How do I define accuracy?
%     This is dataset specific but as I am using the vischema dataset I can measure accuracy as the L1 Loss between the produced memorability map and the ones present in the dataset.

\section{Structure}

% How is the dissertation laid out.

\chapter{Literature Review}

% This should be 6-8 pages long

\section{Memorability}

% Roy Brener An experimental investigation of memory span
% Subjects were presented with units of varying length and were asked to recall them
% The unit could be a sentence, a digit, a geometric shape, etc
% It was found that people are not very good at remembering nonsense sylables or sentences but are much better at remembering digits, consonants, and colours. Remembering geometric designs is placed somewhere in the middle.
% This would seem to indicate that people are not very good at recalling images, however in Nickerson's work we can see that subjects are extremely good at recognising images. 

% R. Nickerson Short-term memory for complex meaningful visual configurations 1965
% Subjects are shown new and old photos and they have to identify if the photo is new or old
% P(Ro/So) = ~0.9
% 95% of all responses were correct
% subjects were more likely to correctly identify a photo as new than they were to correctly identify a photo as old
% Shown 1 image at a time with 50% chance of being old and 50% chance of being new
% 

% L. Standing. Learning 10,000 pictures. In Quarterly Journal of Experimental Psychology, 1973
% ## What is the goal of this paper ##
% Memory capacity and retrieval of pictures and words
% it shows that the capacity for memory of pictures is almost limitless, when measured under appropriate conditions. 
% People are able to recall images incredibly well  
% ## What are its limitations ##
% The experiments don't have many people involved Experment 2 only has 2 subjects and experiment 4, 4.
% He tested normal and vivid pictures
%------------

% F. Brady, T. Konkle, G. A. Alvarez, and A. Oliva. Visual long-term memory has a massive storage capacity for object details. In Proceedings of the National Academy of Sciences, 2008
% This is about how much detail is required to remember an image, rather than if you can just remember it or not
% Are you truly only able to store the gist of items in your long term memory, or are we capable of more?
% It is shown that people are able to store far more information than just the gist in their long term memory, we are able to store fine grained details
% this builds directly onto the world of Standing in 1973
%------------

In R. Breners work \cite{BrenerMemorySpan} subject's abilities to remember a series of units was tested, a unit was different depending on the test, 
\begin{quote}
    In the digit test, for example, each digit was a unit; in the sentence test each sentence was a unit, etc. ... Each nonsense syllable constituted a unit ... Each consonant constituted a unit ... Each [geometric] design constituded a unit. \cite[p.468]{BrenerMemorySpan}
\end{quote}
% This needs a citation ^

It was found that people are not very good at remembering nonsense sylables or sentences but are much better at remembering digits, consonants, and colours. Remembering geometric designs is placed somewhere in the middle. This is of interest because it's very similar to our investigation into what aspects make images memorable. Brener was interested in consonants, colours, geometric designs, etc but the finding that different units can be significantly harder or easier to remember is useful to us. If we swap out units for image properties such as texture, contrast, saturation etc, or even more abstract features that a CNN may recognise, then it should be possible to find how memorabilty is impacted by these.   

The mean score of 5.31 found for geometric designs seems to indicate that people are not great at recalling images, however in R. Nickerson's work \cite{NickersonShortTermMemory} we can see the opposite, subjects are found to be extremely good at recognising images. Nickerson found that subjects shown a series of images are, with great accuracy, able to recall if the photo is one they have seen before or not. An item is referred to as 'new on its first occurrence and old on its second occurrence'\cite[p.156]{NickersonShortTermMemory}, it was found that subjects shown one image at a time, each with equal chance of being old or new, are able to correctly distinguish them with 95\% accuracy. 

Like in \cite{NickersonShortTermMemory},  R. Shephards work \cite{ShepardRecognition} also found that subjects are able to distinguish with great accuracy new and old images, the mean percentage of images correctly identified was 99.7\% after a delay 2 hours, 92.0\% after a delay of 3 days. 87.0\% after a delay of 7 days, and 57.7\% after a delay of 120 days. The introduction of a delay into Nickerson's experiment allows us to see that regardless of whether the image is in our short-term memory (2 hour delay) or our long-term memory (3-120 days) subjects are able to correctly identify an incredible amount of images. 

L. Standing built on the work of Nickerson, Shepard, and Brener in \cite{standing10000pictures}. Four experiments were ran which tested memory capacity and retreival speed for pictures and words, I am interested in the performance related to pictures specifically. He found that the capacity for image recognition from memory is almost limitless, when measured under testing conditions. In Standing's first experiment he tested 'normal' and 'vivid' images. He describes 'normal' images as those that 'may be characterized as resembling a highly variegated collection of competent snapshots'\cite[p.208]{standing10000pictures}, and 'vivid' images are described as 'striking pictures ... with definitely interesting subject matter'\cite[p.208]{standing10000pictures}. Much like how in Brener's work \cite{BrenerMemorySpan} the memorability of an object was variable based on what classification of unit was tested, this work shows that on an even more specific level, the classification of the image into normal or vivid, has an impact on subjects memorabilty, Standing found that the vivid images were more likely to be remembered by subjects.

Lots of work has been done to further build on that of Standing, Nickerson, Brener et al. T. Brady et al. \cite{brady2008visual} performed research into how much information is retained in long term memory and found that subjects are able to remember not just the gist of an image, but also fine grain information such as the state of objects within an image, and that they are able to distinguish between variants of objects shown in images. An example shown is an abacus in two different states, 13/14 subjects were able to distinguish which one they had seen before.

% D. A. Brown, I. Neath, and N. Chater. A temporal ratio model of memory. Psych. Review, 2007.
%An exploration to the extend to which common retrieval princples apply to human memory over short and long timescales
% Traces of items are represented in memory partyly in terms of how long ago they were observed
% Similar mechanisms govern retrieval from memory over many different timescales

T. Konkle et al. build further on the work in \cite{standing10000pictures, brady2008visual} by studying the impact that categorical distinctness has on memorability. This was tested by creating a dataset of images where composed of categories such as tables, cameras, and bread etc. Each category had between 1 and 16 images, a memory test like those performed in \cite{standing10000pictures, brady2008visual}
% check which other memory tests this matches
was performed and the percentage correctly identified was found to decrease as the number of images within a category increased. From this we can see that categorically distinct images are more likely to be remembered. 

Studies by Isola et al. \cite{Isola2011, IsolaParikhTorralbaOliva2011} were performed with the goal of identifying a collection of visual attributes that make images memorable, and to use those to predict the memorability of an image. It was found that properties such as mean saturation, and the log number of objects, has less impact on the memorability score than object statistics. 
%WTF are object statistics?
Categories such as: person, person sitting, and floor, were most helpful image memorability. The categories: ceilings, buildings, and mountains, were least helpful. Their approach was limited by the fact that the object statistics were annotated by hand, this would both make automating the process of determining image memorability impossible, and limit them from finding any abstract properties that helped/hindered memorability.

Khosla et al. \cite{NIPS12_Khosla} built on the work in \cite{Isola2011, IsolaParikhTorralbaOliva2011} by, instead of determining memorability of an entire image, creating a model that discovers memorability maps of individual images without human annotation. These memorability maps are able to distinguish which areas of an image are remembered, forgotten, or hallucinated. Their approach, similar to Isola et al. in \cite{Isola2011, IsolaParikhTorralbaOliva2011}, is limited by the arbitrarily picked list of features that define memorability.

\section{Using Deep Learning to Predict Memorability}

% How do I want this to flow?
%   Deep learning intro? talk about it being an image segmentation problem and something about convolution and learning filters
%   Possible Datasets
%   Auto Encoders (UNET)
%   GANS
%   Diffusion

\subsection{Memorability Datasets}

% Understanding and predicing image memorability at a large scale
%A novel experimental procedure to objectively measure human memory, allowing them to build LaMem, the largest annotated image memorability dataset
% How is it annotated? Memorability Scores in range [0,1]
% Heat maps?
% They show that fine-tuned deep features outperform all other features by a large margin
% They use Hybrid-CNN to classify the categories of objects and scenes. 

Khosla et al. have created a dataset, LaMem \cite{ICCV15_Khosla}, containing 60,000 annotated image and memorability pairs. They use deep learning to predict these memorability maps.

This is similar to the VISCHEMA plus dataset \cite{VischemaPaper}, which I will be working with. In \cite{VischemaPaper} the network output utilises fully connected layers which I believe is unnecessary, and may even hurt performance. More modern computer vision network structures, such as those in \cite{ronneberger2015unet, goodfellow2014generative, ho2020denoising, isola2018imagetoimage, saharia2022palette}, use fully convolutional networks. These maintain spatial locality which allows them to generalise better.

In both \cite{VischemaPaper, ICCV15_Khosla} they present models that predict memorability maps of the images in the datasets presented, they do this by training a classifier to predict the memorability score of an image, and then to split the image into multiple sections which they then predict the memorability of producing a lower resolution memorability map. 

%This might not be relevant for this section
I instead propose to use an Image-to-Image model such as those presented in \cite{ronneberger2015unet, isola2018imagetoimage, ho2020denoising} to take an image from the dataset and predict the memorability map of it from that.

Using an Image-to-Image model we should be able to predict memorability maps of images by training a fully convolutional network across a dataset such as those presented in \cite{VischemaPaper, ICCV15_Khosla}.

\subsection{Image Synthesis Models}

\subsubsection{Autoencoders}

The autoencoder model is composed of 3 parts, an encoder, a bottleneck, and a decoder, these are typically used for compression. The U-Net \cite{ronneberger2015unet} is a variation of an autoencoder which has also found great success in image segmentation within the medical field. It makes use of residual connections between corresponding layers in the encoder and decoder blocks, these allow the model to preserve data for use in the decoding stage. This architecture is the backbone of many image to image models \cite{isola2018imagetoimage,saharia2022palette,dhariwal2021diffusion}, and has seen many extensions \cite{zhou2020unet, Qin_2020}

\subsubsection{Generative Adversarial Networks}
% in 2018? ns on the date, pix2pix was invented, it is a GAN built around the UNet and has also seen great success

% need to introduce GANs, their weaknesses and their benefits

% You typically cant track the progress of a GAN by looking at the loss of the generator or the discriminator, however after running an epoch of training I can compute the loss of the generator across the validation dataset, independent of the discriminator, this allows us to track the progress of the generator and compare it to other methods.


% Maybe here say what a GAN vs a conditional GAN is 

GAN architectures\cite{goodfellow2014generative} use competitive co-evolutionary algorithms where a Generator and a Discriminator compete. The Generator is typically given a latent space vector and uses that to produce an image, the Discriminator has to determine if that image belongs to a given distribution. These networks have seen great success \cite{zhu2020unpaired, karras2019stylebased, isola2018imagetoimage} but are typically unstable and require a lot of parameter fine tuning. They can also suffer from: non-convergence, mode collapse, and diminished gradients \cite{goodfellow2017nips}.

\subsubsection{Diffusion Models}

% talk about significance of diffusion, its benefits and weaknesses
% how it builds ontop of GANs and why they have seen so much success recently

Diffusion based generative models work by learning to iteratively remove Gaussian noise from a sample T times until it produces an image from the training domain. The model was first proposed by Sohl-Dickstein et al. \cite{sohldickstein2015deep}, and has been further developed by Ho et al. \cite{ho2020denoising},  Dhariwal and Nichol \cite{dhariwal2021diffusion}, and Saharia et al. \cite{saharia2022palette}. 

%Diffusion based image generation is inspired by nonequilibrium thermodynamics, it works through a forward diffusion process of adding Gaussian noise to a sample until it has been transformed such that it is no longer distinguishable to Gaussian noise, this should follow a diffusion schedule where noise is applied T times. We then train a model to compute a reverse diffusion process. That is, given a value 1 < t < T, and a sample that has been transformed t times, predict the sample at t-1 transformations. When we want to produce a sample, we create $ x_0 \sim N(0,1) $ and apply reverse diffusion to it T times, following our diffusion schedule. 

Diffusion models, first introduced by J. Sohl-Dickstein et al. \cite{sohldickstein2015deep}, are a machine learning model that work through a forward diffusion process systematically destroying the structure in a data distribution, and the learning of a backwards process to restore the structure. The method uses a Markov chain to convert $ x_t $ into $ x_{t-1} $. Starting with $ x_T $, a sample of Gaussian noise, a generative Markov chain converts this into $ x_0 $, which is a sample from the target data distribution. Because the model only estimates small pertubations of noise, $ x_t $ given $ x_{t-1} $ , rather than an entire transformation from $ x_0 $ to $ x_T $, it is tractable to train. 

The DDPM, a UNet based diffusion model,\cite{ho2020denoising} is capable of producing high quality images and achieves state of the art FID scores across the CIFAR10 dataset. Dhariwal and Nichol \cite{dhariwal2021diffusion} show tweaks that allow for diffusion models to achieve state of the art FID scores across the ImageNet dataset and when used in combination with upsampling diffusion further improve FID scores. They do this by using improvements proposed in \cite{song2022denoising, nichol2021improved, song2021scorebased, brock2019large, karras2019stylebased}. These improvements also reduce the number of noise steps required from thousands to (in some cases) 50. Through the decrease in noise steps they are able to reduce the amount of time that it takes to generate an image. Chen discusses in \cite{chen2023importance} how changing the resolution of an image has an impact on the noise scheduling required, he finds that the optimal scheduler at a smaller resolution may cause under training for higher resolution images. Multiple strategies are proposed to adjust noise scheduling. Firstly, changing the noise schedule functions to those based on cosine or sigmoid, with temparature scaling. Secondly, reducing the input scaling factor from 1 increases the noise levels which destroys more information at the same noise level. They then combine these into a compound noise scheduling strategy.


\chapter{Methodology}

% I dont think this needs to be long, ~2 pages
% This should be from a theoretical perspective
% No need for specific values, just describe what I plan on doing
% Use general ranges of the params

% Motivation?

\section{Requirements Capture}

The VISCHEMA dataset[Vischema paper] contains image to vms mappings. I aim to use a deep learning model to learn a mapping of these images to their corresponding labels. Our model should learn a general understanding of the mapping such that when it is provided with an image that matches our distribution, it can accurately create a VMS label for it. Our model will need to learn to create accurate mappings and we can test that through a loss function, such as L1, and through qualitative analysis.

\section{Motivation}

As standard backpropagation, GANs, and diffusion all produce images in different ways I think it would be interesting to compare how the 3 of them perform when asked the same task. I will run three experiments: I will train a UNet, a GAN, and a diffusion model to produce a VMS label given an image from the dataset. I hope to be able to find the strengths and weaknesses of each of these systems in this application.

For each system I will experiment with network parameters and training hyperparameters to fine tune models that produce the best output. I will experiment with training the models to produce the sum of the images and the VMS labels, and with training the models to produce just the VMS label. In the former the VMS label can be calculated from the output. 

In each experiment I will automate a system to: test the use of different values for the number of layers and channels in each network, vary the normalisation method, optimisation function, learning rate, and batch size to find the best training environment and model parameters. As it is not be feesible to test every combination of these variables, I will employ a training strategy to test every variation of a single parameter/hyperparameter while keeping the rest constant, pick the best scoring variation of that parameter, and move on to the next. I will start with choices that I estimate to be good, as this should only improve on them. This will bring the size our search space down by an order of magnitude, however we won't be exploring the entire search space and will potentially miss out on good values. This will be especially necessary in experiment 2 as training within a GAN is typically unstable [papers showing instability in GANs]. In experiment 3 I will also vary the noise scheduler and the beta values.

\section{Experiment 1}
% What is the name for this type of training?

\textbf{Model}: This network is a UNet autoencoder that takes as input a 64x64x3 tensor of floating point values in the range [-1,1], these are the images in our dataset.
% double check this input range
It outputs a 64x64x3 tensor of floating point values in the range [-1,1], these are our label estimates.
% double check this output range
This model can be described as: \[ L_{pred} = model(I) \]

\textbf{Model Loss}: This is simply the L1 loss between $L_{pred}$ and $L_{real}$, describing how closely the output of our network matches the corresponding label. At each epoch we will calculate this over the training dataset, use that for backpropagation, and then calculate it across the validaton dataset to test how well we have generalised.
\[ L = L1( model(I), L_{real} ) \]

\textbf{Training Strategy}: Please see algorithm \ref{ALG:Autoencoder} for the training loop.

\begin{algorithm}
\caption{UNet Autoencoder Training Strategy}\label{ALG:Autoencoder}
\begin{algorithmic}[1]
\For{$\texttt{every epoch}$}
\For{$I, L_{real} \texttt{ in training dataloader}$}
\State
\State $L_{pred} = model(I)$
\State $loss = L1( L_{pred}, L_{real} ) $
\State $\texttt{Update weights of model with backpropagation}$
\State
\EndFor
\EndFor
\end{algorithmic}
\end{algorithm}

\section{Experiment 2}

I will train a conditional generative adversarial network to generate images of VMS maps given images from the VISCHEMA dataset. I will use an adapted Pix2Pix network \cite{isola2018imagetoimage}, making tweaks that I think will increase performance. 
Pix2Pix by Isola et al.\cite{isola2018imagetoimage}, is an image-to-image conditional generative adversarial model based on the UNet \cite{ronneberger2015unet}. Designed to translate images from one style into another. In \cite{arjovsky2017wasserstein} M. Arjovsky et al. introduce a GAN variant based on the Wasserstein distance between the output distribution and the image distribution, the benefit of this is that the Wasserstein distance is continuous and differentiable almost everywhere, reducing the risk of dimminishing gradients. In \cite{pix2pixwasserstein} N. Makow investigated the use of Wasserstein Distance in the Pix2Pix model but unfortunately found that it does not perform much better, I would still like to experiment with it as VISCHEMA plus is different from any dataset used in \cite{isola2018imagetoimage} and as Makow states they were unable to perform a complete hyperparameter search, meaning that its possible we could achieve greater results than vanilla Pix2Pix.
I will be adapting the following Pytorch implementation \cite{PytorchPix2Pix}. 

Lots of work has gone into making GAN models more stable and I will use these findings in my own models. In \cite{brock2019large} A. Brock et al. found that when 'increasing the batch size by a factor of 8 ... models reach better final performance in fewer iterations, but become unstable and undergo complete training collapse'. Because of this I will experiment with early stopping and low batch sizes.  In \cite{zhao2020differentiable} Z. Shengyu et al. show how using differentiable augmentation on your images increases the quality of the outputted images. With a dataset with as few as 100 images they are able to produce high quality outputs, this is useful to us because the VISCHEMA plus dataset only has 1280 training images, which typically wouldn't train very well on a GAN.

% https://www.youtube.com/watch?v=J97EM3Clyys
% https://github.com/mit-han-lab/data-efficient-gans/blob/master/DiffAugment-stylegan2-pytorch/DiffAugment_pytorch.py
% horizontal flips are a good application on the real images
% You could augment the real images -> this will cause the generator to learn to generate augments, this is bad for augments such as cutout
% You could augment the real images and the fake images when the discriminator sees them, the generator will have an easier time fooling the discriminator
% Solution: Augment real and fake images for the discriminator and the generator
% Need to be able to differentiate the augmentation so that we can update the generators weights using gradient descent
% How do we make our augmentations differentiable?
% We provide a torch tensor that represents the images as input, then we randomly sample parameters related to augmentation. e.g. translation amount or cutout coords. Apply this augmentation to the input tensor
% since it is all done using torch operations, the computational graph will be computed for these functions
% Kornia can do this for us.
% pass the augmentation module, kornia, into the discriminator. Apply augments on the input to the discriminator first but only when training 

%https://cs230.stanford.edu/projects_spring_2018/reports/8289943.pdf 


\textbf{Generator}: This network takes as input a 64x64x3 tensor of floating point values and outputs a 64x64x3 tensor of floating point values in the range [-1,1]. This model can be described as: \[ L_{f} = G(I) \]

\textbf{Discriminator}: This network takes as input a 64x64x6 tensor of floating point values. Channels 1,2, and 3 store the image and channels 4,5, and 6 are its corresponding label, either real or generated. It outputs a 4x4x1 tensor of boolean values. Each value in this output represents a 16x16x6 region of the input.

The loss for each network is computed as described in [pix2pix paper], across an entire batch of images and then the weights are adjusted with backpropagation. The optimiser used is one of the hyperparameters that we will search for.

\textbf{Generator Loss}: The generator loss describes how well it can trick the discriminator, and how closely its output matches the real label for the given image.
\[ L = MSE( D(L_{f}, I), 1 ) + L1(L_{f}, L_{r}) \]

\textbf{Discriminator Loss}: The discriminator loss describes how accurately it is able to predict, given a label and an image, if the label is real or not.
\[ L = 0.5 \times  ( MSE( D( L_{f}, I ), 0) + MSE( D( L_{r}, I ), 1) ) \]

Because these two loss values are relative to the performance of eachother they can't be used to see if our generator has converged on a solution. Therefore it is necessary, at each epoch, to also compute the L1 loss of the fake labels and real labels across the training and the validation dataset, as these values are independent of the discriminator. This will inform allow us to see if the generator is overfitting, underfitting, or training well. 

\textbf{Training Strategy}: Please see algorithm \ref{ALG:GAN} for the training loop.

\begin{algorithm}
\caption{GAN Training Strategy}\label{ALG:GAN}
\begin{algorithmic}[1]
\For{$\texttt{every epoch}$}
\For{$I, L_{r} \texttt{ in training dataloader}$}
\State
\State $L_{f} = G(I)$
\State
\State $pred_fake = D(L_{f}, I) $
\State $loss_G = MSE( pred_fake, 1 ) + L1(L_{f}, L_{r}) $
\State $\texttt{Update weights of G with backpropagation}$
\State
\State $pred_real = D(L_{r}, I)$
\State $loss_D = 0.5 \times  ( MSE( pred_fake, 0) + MSE( pred_real, 1) ) $
\State $\texttt{Update weights of D with backpropagation}$
\State
\EndFor
\EndFor
\end{algorithmic}
\end{algorithm}

\section{Experiment 3}

I will adapt a PyTorch implementation of Palette \cite{JanspiryPalette} and train it to predict labels from the VISCHEMA dataset. Palette is a versitile conditional image to image diffusion model proposed by Saharia et al. \cite{saharia2022palette} that is able to uncrop, inpaint, colorize, and remove JPEG artifacts from images. Because of this versitility I think it could learn to predict VMS maps from images. The conditional image passed to the model will be the image from the dataset, the ground truth will be the label. This network takes as input a 128x128x6 tensor of floating point values, this is the concatenation of an image and the noise added to the ground truth after $t$ steps. It outputs a 128x128x3 tensor of floating point values, this is the models estimate of the noise added at time $t-1$.
This model can be described as \[ noise_{pred} = model(I, t) \]

\textbf{Model Loss}: The model loss describes how well it can estimate the noise added to an image between time steps $t-1$ and t, it is computed across an entire batch of images and then the model weights are adjusted with backpropagation.

\[ L = MSE( model(I, t), noise_{real} ) \]

% This but for diffusion
The loss calculated for backpropagation is relative to the performance in estimating noise added, not for the performance when calculating VMS labels. Because of this I will also have to calculate the L1 loss of the generated labels against the real labels. This will take a lot of computational time so I will do it at the end of training. I will be able to see if our model has converged by observing the backpropagation loss. 

\textbf{Training Strategy}: Please see algorithm \ref{ALG:Diffusion} for the training loop.

% Maybe theres a better way to write this?
% Look at the papers and see if its there
\begin{algorithm}
\caption{Diffusion Model Training Strategy}\label{ALG:Diffusion}
\begin{algorithmic}[1]
\For{$\texttt{every epoch}$}
\For{$I, L \texttt{ in training dataloader}$}
\State
\State $t = random(0, noise_steps)$
\State $noise_{real}, L_{noisy} = noise\_image(L, t)$ \Comment{Apply t steps of noise and return noise added at step t}
\State
\State $input = concatenation(I, L_{noisy})$
\State $noise_{pred} = model(I, t)$
\State $loss = MSE( noise_{pred}, Noise_{real} )$
\State $\texttt{Update weights of model with backpropagation}$
\EndFor
\EndFor
\end{algorithmic}
\end{algorithm}

\section{Results Analysis/Testing}

After performing all 3 experiments I will compare the results across the validation dataset qualitatively and using the L1 loss. The L1 loss will tell me how statistically close the outputs are but through qualitative analysis I can see if the outputs have cheated. 
% need to explain cheating better but basically the large blobs
I can also compare the L1 loss across the training and validation sets to see if any models have generalised well, if the loss across the training dataset is much smaller/larger than the validation dataset then it will imply that the model has become overfit/underfit respectively. Ideally they should be similar. 


\chapter{Experimental Results}

% Approx 10-15 pages

% add plots and tables of results and explain them
% important facts in text
% compare loss functions
% explain dataset limitations
%    solved with gan
%    in experiments?

\section{Experimental Environment}

I have implement the experiments in Python using the PyTorch machine learning framework, however the methodology that I describe should produce the same results in any programming language or framework.

This model was trained on a computer using an RTX 3070 with 8GB of VRAM and an AMD Ryzen 3600 with 32GB of system RAM.
Testing the hyperparameter options for experiment 1 took approximately X days and I trained the final model over the course of Y hours
% FILL IN X AND Y
Testing the hyperparameter options for experiment 2 took approximately 3 days and I trained my final model over the course of 5 hours.
% Need to double check that  
Testing the hyperparameter options for experiment 1 took approximately X days and I trained the final model over the course of Y hours
% FILL IN X AND Y

All experiments were performed with differentiable augmentation as described in \cite{zhao2020differentiable}. I performed translation and cutout, each with a 90\% chance.
% THAT PERCENT MAY CHANGE

\section{Hyperparameter tuning}

For experiments 1 and 2 I automated the process of exploring the parameter and hyperparameter search space. In my search I varied the following parameters: The normalisation layer used, the channel layouts used, the optimiser used, the learning rates for the optimiser, and, if the Adam optimiser was used, the beta values. Because training a diffusion model is far more computationally expensive than training an Autoencoder or a GAN I was unable to automate the process of tuning the hyperparameters in experiment 3. Training a GAN model on the VISCHEMA dataset would typically take around 100 minutes, but training a diffusion model would take around 20 hours due to the greater number of epochs required.


In experiment 2 I allowed for the generator and discriminator models to use different normalisation functions, optimisers, and learning rates. 


Exploring this search space exhaustively is unfortunately not feesible, there are over 5000 different combinations possible, and if I tested each combination once for 100 epochs then it would take approximately 300 days to test per experiment.
% double check these numbers
% should I describe this in my methodology or above?
% how long does it take to train an epoch?
Instead I will have to explore a subset of this search space. For each experiment I estimated some good default parameter and hyperparameter values, by varying these values I can lower the scope of the search space to 
% maybe I should get an exact value here instead of 100
approximately 100 combinations, which took \(\sim\)3 days per experiment to test. Unfortunately this does mean that not every combination has been tested, however we should achieve a good approximation of the best parameters and hyperparameters.


% Should this go at the end?
In experiment 3 I was only able to explore a small range of possible hyperaparameters and as such my results do not emulate the recent success found in diffusion based image generation.
% papers on the success of diffusion.
However, with a greater number of computing resources it may be possible to do so.

\section{Experiment 1}

In this experiment I trained a UNet to predict the VMS maps of images, I trained the model using backpropagation of the L1 loss of the model output given an image, and the corresponding label. 

The best parameters and hyperparameters that I found for my model were the following:
% Input best hyperparameters

My training graphs:
% Input training graphs

After X training epochs I was able to achieve an L1 loss across my validation dataset of Y.
% Input X and Y
I was also able to achieve a FID score of Z
% Input Z

A sample of output images:
% Output images

When performed without differentiable augmentation the results looked like this:
% Images.

FID score of Z, L1 Loss of Y.
% Input Z, Y

\section{Experiment 2}

I explored the following parameter and hyperparameter options:

\textbf{Normalisation layers}: Batch Normalisation, Instance Norm

\textbf{Channel layouts}:
I iterated over generator encoder and decoder, and discriminator channel layouts of following form, with values of $c$ from \{32,64,50,100\}:
\begin{itemize}
\item Encoder: (3, $c$, 2$c$, 4$c$, 8$c$, 16$c$), 
\item Decoder: (16$c$, 8$c$, 4$c$, 2$c$, $c$),
\item Discriminator: (6, $c$, 2$c$, 4$c$, 8$c$, 16$c$).
\end{itemize}

\textbf{Optimisers}:
\begin{itemize}
\item SGD, using the following learning rates: 0.005, 0.01, 0.02,
\item Adam, using the following learning rates: 0.0005, 0.001, 0.002, and using the following betas values: (0.9, 0.999), (0, 0.999), (0.5, 0.999),
\item Adadelta, using the following learning rates: 0.5, 1, 2.
\end{itemize}

\emph{Note: I didn't need to specify default values for the normalisation layer as these were the first variables I tested.}

\textbf{Default Generator Parameters and Hyperparameters}: 
\begin{itemize}
\item $C$ = 64
\item Optimiser: Adam, betas = (0.9, 0.999)
\item Learning Rate: 0.001
\end{itemize}
\textbf{Default Discriminator Parameters and Hyperparameters}:
\begin{itemize}
\item $C$ = 64
\item Optimiser: Adam, betas = (0.9, 0.999)
\item Learning Rate: 0.001
\end{itemize}
I iterated over each different parameter and hyperparameter and varied each one sequentially. I tested each combination for 100 epochs and used the L1 loss across the validation dataset as the score, if any new parameter options gave a better L1 score the it would become the default used going forward. This meant that I only had to iterate \(\sim\)100 combinations. I found the following best options.

\textbf{Best Generator}: Batch Normalisation, $C$ = 32, SGD Optimiser with a learning rate of 0.01.

\textbf{Best Discriminator}: Batch Normalisation, $c$ = 100, Adam optimiser with betas = (0.9, 0.999) and a learning rate of 0.001.

With this combination we achieved a loss of \(\sim\)1.02 across the validation dataset. I found other good results using similar combinations. Using the Adatelta optimiser for the generator and the Adam optimiser for the discriminator achieved a loss of \(\sim\)1.03 across the validation dataset. Using the Adam optimiser for both the generator and discriminator achieved a loss of \(\sim\)1.04 across the validation dataset.

% Insert graph of the best combinations results here
% Insert images

FID score of Z, L1 Score of Y
% Input Z, Y

Here are the images generated when I dont use differentiable augmentation
% Insert images
% Insert training graphs

FID score of Z, L1 Score of Y
% Input Z, Y

\textbf{Wasserstein GAN}: Using a Wasserstein image-to-image GAN as described in \cite{pix2pixwasserstein} I was able to achieve a best score of
% Insert score here.
% Insert images.

FID score of Z, L1 Score of Y
% Input Z, Y

Here are the images generated when I dont use differentiable augmentation
% Insert images
% Insert training graphs

FID score of Z, L1 Score of Y
% Input Z, Y

\section{Experiment 3}



\section{Results, Evaluation Metrics, and Analysis}

a graph of the train/val loss over time across the different experiments


a graph of the FID score at the end of the different experiments


a selection of the images with the different training methods


Some discussion about them.


\chapter{Conclusion}

% Approx 1-2 pages
% how it went
% future work better generator better result
% applications

\printbibliography

\end{document}